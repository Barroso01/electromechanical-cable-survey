\label{sec:INTRO}
Precise and cost-efficient simulation of AC cable systems is crucial in modern electrical engineering. Fast and reliable simulation can improve the design and development process, allowing engineers to explore, evaluate and prototype more advanced and safe electrical wires. However, as these currents flow through cables, complex electromagnetic phenomena emerges such as skin effect, electromagnetic interference and proximity effect. Failing to understand the complex analysis required leads to costly or inaccurate simulations. 
\\

This work presents the theoretical implications of AC systems and implements accurate and cost-efficient simulation model. The analysis primarily revolves around three-phase alternating currents, commonly encountered in electrical systems, with the standard frequency of 60 Hertz and current values of 5,000 amperes. Results presented in this work are based on a single nuclei, copper wire with a caliber of 400 kcmil. In compliance with the guidelines outlined by the National Electric Code, the system is designed to accommodate a 5,000 ampere load using 15 cables.
\\

%%%%%%%%%%%%%%%%%%%%%%%%%%%%%%%%%%%%%%%%%%%%%%%%%%%%%%%%%%%%%%%%%%%%%%%%%%%%%%%%%%%%%%%%%%
\section{Variables}
To facilitate a comprehensive understanding of the model, the work presents a list of key variables that will be central to our analysis. Furthermore, thought sections \ref{sec:Conducting Matter and Currents} and \ref{sec:Quasistatic Fields} we will use the numerical values for such variables to present theoretical calculations. It is recommended to take a moment to familiarize with the critical variables. 
\\
\begin{table}[H]
\resizebox{\columnwidth}{!}{ % Scale to column width
\begin{tabularx}{\linewidth}{|c|c|X|}
\hline
\textbf{Name} & \textbf{Units} & \textbf{Value}\\
\hline
Current per cable ($I_0$) & [A] & 333.3 \\
\hline
Frequency ($w$) & [Hz] & $60 * (2\pi)$ \\
\hline
Radius (R) & [m] & 0.00897 \\
\hline
Length of cable & [m] & 1  \\
\hline
Electron mass ($m$) & [kg] & $9.1 * 10^{-31}$ \\
\hline
Electron charge ($q_e$) & [C] & $1.6021766 * 10^{-19}$ \\
\hline
Relaxation Time ($\tau$) & [s] & $2.48* 10^{-14}$ \\
\hline
Electron Density (n)& [$m^{-3}$] & $8.5 * 10^{28}$ \\
\hline
Conductivity ($\sigma$) & [$(\Omega m)^{-1}$] & $5.95*10^7$ \\
\hline
Permittivity ($\varepsilon$) &[F/m] & $ 1.476 * 10^{-6}$ \\
\hline
Permittivity free space ($\varepsilon_0$) & [F/m] & $8.8541878128 * 10^{-12}$ \\
\hline
Permeability ($\mu$) & [H/m] & $4\pi*10^{-7}$ \\
\hline
\end{tabularx}
} % Close fit
\caption{Critical Variables. This table presents key parameters related to copper conductors used in the study. The units specified for each parameter are shown in square brackets. }
\end{table}

The variables of permittivity and relaxation time are calculated with Drude's model of conductivity at low frequencies ($w < 10^{11}$) with no loss of generality. Furthermore, in section \ref{sec:RESULTS} the reader is presented with the calculations that take into account the experimental value of conductivity. This is done in order to ensure the accuracy and precision of the model. However, before diving into the details, the work proceeds to explain some basic notions to get all readers into context.
\\


