This work successfully developed an accurate simulation model. Furthermore, by analyzing the theoretical implications of AC systems, valuable insights were obtained regarding the behavior of these systems and their associated electromagnetic phenomena.
\\

The implementation of the Python model allowed for fast and reliable simulations, improving the design and development process of electrical wires. By utilizing this simulation model, engineers can explore, evaluate, and prototype more advanced and safe electrical wires, saving time and resources compared to traditional trial-and-error methods.
\\

The results obtained through the simulation model demonstrated a high level of accuracy when compared to experimental data. The heat flux in the cable system was accurately predicted, validating the effectiveness of the simulation model in capturing the thermal behavior of AC cable systems. These findings contribute to the advancement of simulation techniques, enabling engineers to make informed decisions and optimize the design of AC cable systems for enhanced performance and efficiency.
\\

In summary, this study has demonstrated the significance of precise and cost-efficient simulation in AC cable systems. By integrating theoretical considerations, implementing an accurate Python model, and validating the results with experimental data, this research provides valuable insights for engineers and researchers in the field. The developed simulation model serves as a powerful tool for optimizing the design and performance of AC cable systems, ultimately contributing to the advancement of electrical engineering practices.